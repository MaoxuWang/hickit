\documentclass[10pt]{article}

\usepackage{amsmath}
\newcommand{\norm}[1]{\left\lVert#1\right\rVert}

\begin{document}

For two constrained particles at coordinates $\vec{x}_1$ and $\vec{x}_2$,
respectively, let
$$\vec{r}=\vec{x}_1-\vec{x}_2$$
$$r=\norm{\vec{r}}=\sqrt{r_1^2+r_2^2+r_3^2}$$
$$\hat{r}=\nabla r=\vec{r}/r$$

The general form of potential energy $U(r)$ between the two particles is
\begin{equation*}
U(r)=\left\{\begin{array}{ll}
k(d_1-r)^2 & (r < d_1) \\
0 & (d_1\le r\le d_2) \\
k(r-d_2)^2 & (d_2 < r \le d_3) \\
k[c_1(r-d_3) + c_2/(r-d_2)] & (r > d_3)
\end{array}\right.
\end{equation*}
The force between the two particles is $\vec{F}(\vec{r})=\nabla
U(r)=U'(r)\cdot\nabla r$, which is
\[
\vec{F}(\vec{r})=\left\{\begin{array}{ll}
-2k(d_1-r)\cdot\hat{r} & (r < d_1) \\
0 & (d_1\le r\le d_2) \\
2k(r-d_2)\cdot\hat{r} & (d_2 < r \le d_3) \\
k[c_1-c_2/(r-d_2)^2]\cdot\hat{r} & (r > d_3)
\end{array}\right.
\]
To find $c_1$ and $c_2$, we require $U(r)$ and $U'(r)$ continuous at
$r=d_3$. This gives $c_1=3(d_3-d_2)$ and $c_2=(d_3-d_2)^3$. Note that $U''(r)$
is also continuous at $r=d_3$.

\end{document}
